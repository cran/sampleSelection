\documentclass[article,nojss]{jss}
\usepackage[utf8]{inputenc}
\usepackage{amsmath,amssymb,bbm}
%% need no \usepackage{Sweave.sty}

\usepackage{csquotes}
\MakeOuterQuote{§}

\newcommand{\dnorm}{\phi}% normal density
\newcommand{\loglik}{\ell}% log likelihood
\newcommand*{\mat}[1]{\mathbf{#1}}% Matrix
\newcommand{\pnorm}{\Phi}% normal distribution function
\renewcommand*{\vec}[1]{\boldsymbol{#1}}% vector

\author{Arne Henningsen\\University of Copenhagen}
\Plainauthor{Arne Henningsen}

\title{Estimating Censored Regression Models in \proglang{R} using the \pkg{sampleSelection} Package}
\Plaintitle{Estimating Censored Regression Models Models in R using the sampleSelection Package}

\Abstract{
We show how censored regression models
(including standard Tobit models)
can be estimated in \proglang{R}
using the add-on package \pkg{sampleSelection}.
}

\Keywords{censored regression, Tobit, econometrics, \proglang{R}}
\Plainkeywords{censored regression, Tobit, econometrics, R}

\Address{
  Arne Henningsen\\
  Institute of Food and Resource Economics\\
  University of Copenhagen\\
  1958 Frederiksberg, Denmark\\
  E-mail: \email{arne.henningsen@gmail.com}\\
  URL: \url{http://www.arne-henningsen.name/}
}

\begin{document}
% initialisation stuff

%\VignetteIndexEntry{Censored Regression Models}
%\VignetteDepends{sampleSelection,AER}
%\VignetteKeywords{censored regression, Tobit model, econometrics, R}
%\VignettePackage{sampleSelection}

\section{Introduction}
\label{sec:intro}

In many statistical analyses of individual data,
the dependent variable is censored,
e.g.\ the number of hours worked,
the number of extramarital affairs,
the number of arrests after release from prison,
purchases of durable goods,
or expenditures on various commodity groups
\citep[p.~869]{greene08}.
If the dependent variable is censored (e.g.\ zero in the above examples)
for a significant fraction of the observations,
parameter estimates obtained by conventional regression methods (e.g.\ OLS)
are biased.
Consistent estimates can be obtained by the method
proposed by \citet{tobin58}.
This approach is usually called §Tobit§ model
and is a special case of the more general censored regression model.

This paper briefly explains the censored regression model,
describes function \code{tobit} of the \proglang{R} package
\pkg{sampleSelection},
and demonstrates how this function can be used
to estimate censored regression models.

There are also some other functions for estimating censored regression models
available in \proglang{R}.
For instance function \code{tobit} from the \pkg{AER} package
\citep{kleiber08a,r-aer-1.1}
and function \code{cenmle} from the \pkg{NADA} package
are front ends to the \code{survreg} function from the \pkg{survival} package.
Function \code{tobit} from the \pkg{VGAM} package
estimates the censored regression model
by using its own maximum likelihood routine.
Function \code{MCMCtobit} from the \pkg{MCMCpack} package
uses the Bayesian Markov Chain Monte Carlo (MCMC) method
to estimate censored regression models.


\section{Censored regression model for cross-sectional data}
\label{sec:censored}


\subsection{Standard Tobit model}

In the standard Tobit model \citep{tobin58},
we have a dependent variable $y$ that is left-censored at zero:
\begin{align}
y_i^* &= x_i ' \beta + \varepsilon_i\\
y_i &=
   \begin{cases}
   0     & \text{if } y_i^* \leq 0\\
   y_i^* & \text{if } y_i^* > 0
   \end{cases}
\end{align}
Here the subscript $i = 1, \ldots , N$ indicates the observation,
$y_i^*$ is an unobserved (§latent§) variable,
$x_i$ is a vector of explanatory variables,
$\beta$ is a vector of unknown parameters, and
$\varepsilon_i$ is an error term.


\subsection{Censored regression model}

The censored regression model is a generalisation of the standard Tobit model.
The dependent variable can be either left-censored, right-censored,
or both left-censored and right-censored,
where the lower and/or upper limit of the dependent variable can be any number:
\begin{align}
y_i^* &= x_i ' \beta + \varepsilon_i\\
y_i &=
   \begin{cases}
   a     & \text{if } y_i^* \leq a\\
   y_i^* & \text{if } a < y_i^* < b\\
   b     & \text{if } y_i^* \geq b
   \end{cases}
\end{align}
Here $a$ is the lower limit and $b$ is the upper limit
of the dependent variable.
If $a = -\infty$ or $b = \infty$,
the dependent variable is not left-censored or right-censored,
respectively.


\subsection{Estimation Method}

Censored regression models (including the standard Tobit model)
are usually estimated by the Maximum Likelihood (ML) method.
Assuming that the error term $\varepsilon$ follows a normal distribution
with mean $0$ and variance $\sigma^2$,
the log-likelihood function is
\begin{align}
\log L  = 
   \sum_{i = 1}^N \bigg[ \;&
      I_i^a \log \pnorm \left( \frac{ a - x_i ' \beta }{ \sigma } \right)
      + I_i^b \log \pnorm \left( \frac{ x_i ' \beta - b }{ \sigma } \right)
      \label{eq:logLik}\\
   & + \left( 1 - I_i^a - I_i^b \right)
      \left(
         \log \dnorm \left( \frac{ y_i - x_i ' \beta }{ \sigma } \right)
         - \log \sigma
      \right)
   \bigg], \nonumber
\end{align}
where $\dnorm(.)$ and  $\pnorm(.)$ denote the probability density function
and the cumulative distribution function, repectively,
of the standard normal distribution,
and $I_i^a$ and $I_i^b$ are indicator functions with
\begin{align}
I_i^a & =
   \begin{cases}
   1 & \text{if } y_i = a\\
   0 & \text{if } y_i > a
   \end{cases}\\
I_i^b & =
   \begin{cases}
   1 & \text{if } y_i = b\\
   0 & \text{if } y_i < b
   \end{cases}
\end{align}
The log-likelihood function of the censored regression model~(\ref{eq:logLik})
can be maximised with respect to the parameter vector $( \beta' , \sigma )'$
using standard non-linear optimisation algorithms.


\subsection[Implementation in function tobit]{Implementation in function \code{tobit}}

Censored regression models can be estimated in \proglang{R}
with function \code{tobit},
which is available in the \pkg{sampleSelection} package
\citep{toomet08}.
The most important steps done by the \code{tobit} function are:
\begin{enumerate}
\item perform basic checks on the arguments provided by the user
\item prepare the data for the estimation,
   i.e.\ the vector of the dependent variable $y = ( y_1 , \ldots , y_N )'$
   and the matrix of the regressors $X = ( x_1 ' , \ldots , x_N ' )'$
\item obtain initial values of the parameters $\beta$ and $\sigma$
   from an OLS estimation
   (if no initial values are provided by the user)
\item define the log-likelihood function
   (a function that returns the log-likelihood value
   given the vector of parameters $( \beta' , \sigma )'$
\item define a function that calculates the gradients
   of the log-likelihood function given the vector of parameters\footnote{
      The gradients of the log-likelihood function are presented
      in appendix~\ref{sec:logLikGrad}.}
\item call function \code{maxLik} of the \pkg{maxLik} package
   \citep{r-maxlik-0.7} for the maximisation of the likelihood function
\item add class \code{"tobit"} to the returned object
\end{enumerate}


\subsection[Using function tobit]{Using function \code{tobit}}

Before function \code{tobit} can be used,
the \pkg{sampleSelection} package \citep{toomet08}
must be loaded:
\begin{Schunk}
\begin{Sinput}
R> library("sampleSelection")
\end{Sinput}
\end{Schunk}
The first argument of function \code{tobit} is \code{formula}.
It is the only mandatory argument and
must provide a symbolic description of the model to be fitted.
The optional argument \code{data} can be used to provide
a data set (\code{data.frame})
that contains the variables used in the estimation.
We demonstrate the usage of \code{tobit} by replicating an example
given in \citet[p.~142]{kleiber08a}.
The data used in this example are available in the data set \code{Affairs}
that is included in the \proglang{R} package \pkg{AER}
\citep{kleiber08a,r-aer-1.1}.
This data set can be loaded by the following command:
\begin{Schunk}
\begin{Sinput}
R> data("Affairs", package = "AER")
\end{Sinput}
\end{Schunk}
In the example of \citet[p.~142]{kleiber08a},
the number of a person's extramarital sexual intercourses (§affairs§)
in the past year
is regressed on the person's age, number of years married,
religiousness, occupation, and own rating of the marriage.
The dependent variable is left-censored at zero and not right-censored.
Hence, this is a standard Tobit model.
It can be estimated by following command:
\begin{Schunk}
\begin{Sinput}
R> tobitResult <- tobit(affairs ~ age + yearsmarried + religiousness + 
+     occupation + rating, data = Affairs)
\end{Sinput}
\end{Schunk}
Detailed estimation results can be obtained by using the \code{summary} method.
\begin{Schunk}
\begin{Sinput}
R> summary(tobitResult)
\end{Sinput}
\begin{Soutput}
--------------------------------------------
Maximum Likelihood estimation
Newton-Raphson maximisation, 7 iterations
Return code 1: gradient close to zero
Log-Likelihood: -705.5762 
7  free parameters
Estimates:
               Estimate Std. error t value   Pr(> t)    
(Intercept)    8.174197   2.741446  2.9817  0.002866 ** 
age           -0.179333   0.079093 -2.2674  0.023368 *  
yearsmarried   0.554142   0.134518  4.1195 3.798e-05 ***
religiousness -1.686220   0.403752 -4.1764 2.962e-05 ***
occupation     0.326053   0.254425  1.2815  0.200007    
rating        -2.284973   0.407828 -5.6028 2.109e-08 ***
logSigma       2.109859   0.067098 31.4444 < 2.2e-16 ***
---
Signif. codes:  0 ‘***’ 0.001 ‘**’ 0.01 ‘*’ 0.05 ‘.’ 0.1 ‘ ’ 1 
--------------------------------------------
\end{Soutput}
\end{Schunk}
In case of a censored regression with left-censoring not at zero
and/or right-censoring,
arguments \code{left} (defaults to zero) and
\code{right} (defaults to infinity)
can be used to specify the limits of the dependent variable.
A lower (left) limit of minus infinity (\code{-Inf})
and an upper (right) limit of infinity (\code{Inf})
indicate that there is no left-censoring and right-censoring, respectively.
For instance, minus the number of extramarital sexual intercourses
is not left-censored but right-censored at zero.
The same model as above
but with the negative number of affairs as the dependent variable
can be estimated by
\begin{Schunk}
\begin{Sinput}
R> tobitResultMinus <- tobit(I(-affairs) ~ age + yearsmarried + 
+     religiousness + occupation + rating, left = -Inf, right = 0, 
+     data = Affairs)
\end{Sinput}
\end{Schunk}
This estimation returns $\beta$ parameters that have the opposite sign
of the $\beta$  parameters estimated in the original model,
but the (logarithmised) standard error of the residuals remains unchanged.
\begin{Schunk}
\begin{Sinput}
R> cbind(coef(tobitResult), coef(tobitResultMinus))
\end{Sinput}
\begin{Soutput}
                    [,1]       [,2]
(Intercept)    8.1741974 -8.1741974
age           -0.1793326  0.1793326
yearsmarried   0.5541418 -0.5541418
religiousness -1.6862205  1.6862205
occupation     0.3260532 -0.3260532
rating        -2.2849727  2.2849727
logSigma       2.1098592  2.1098592
\end{Soutput}
\end{Schunk}



\section{Censored regression model for panel data}
\label{sec:panel}

\subsection{Specification}

The censored regression model for panel data
with individual specific effects has following specification:
\begin{align}
y_{it}^* &= x_{it} ' \beta + \varepsilon_{it} =  x_{it} ' \beta + \mu_{i} + \nu_{it} \\
y_{it} &=
   \begin{cases}
   a        & \text{if } y_{it}^* \leq a\\
   y_{it}^* & \text{if } a < y_{it}^* < b\\
   b        & \text{if } y_{it}^* \geq b
   \end{cases}
\end{align}
Here the subscript $i = 1, \ldots , N$ indicates the individual,
subscript $t = 1, \ldots , T_i$ indicates the time period,
$T_i$ is the number of time periods observed for the $i$th individual,
$\mu_i$ is a time-invariant individual specific effect,
and $\nu_{it}$ is the remaining disturbance.


\subsection{Fixed effects}

In contrast to linear panel data models,
we cannot get rid of the individual effects by the within transformation.


\subsection{Random effects}

If the individual specific effects $\mu_i$ are independent
of the regressors $x_{it}$,
the parameters can be consistently estimated with a random effects model.
Assuming
that the individual specific effects $\mu$ follow a normal distribution
with mean $0$ and variance $\sigma_\mu^2$,
the remaining disturbance $\nu$ follows a normal distribution
with mean $0$ and variance $\sigma_\nu^2$, and
$\mu$ and $\nu$ are independent,
the likelihood contribution of a single individual $i$ is
\begin{align}
L_i  = \int_{-\infty}^\infty
   \Bigg\{ \prod_{t=1}^{T_i} \;&
      \left[
         \pnorm \left( \frac{ a - x_{it} ' \beta - \mu_i }{ \sigma_\nu } \right)
      \right]^{I_{it}^a}
      \left[
         \pnorm \left( \frac{ x_{it} ' \beta + \mu_i - b }{ \sigma_\nu } \right)
      \right]^{I_{it}^b}
      \label{eq:likRandom}\\
      & \left[
         \frac{1}{\sigma_\nu}
         \dnorm \left( \frac{ y_{it} - x_{it} ' \beta - \mu_i }{ \sigma_\nu } \right)
      \right]^{\left( 1 - I_{it}^a - I_{it}^b \right)}
   \Bigg\} \; \dnorm \left( \frac{\mu_i}{\sigma_\mu} \right) \; d \mu_i
      \nonumber
\end{align}
and the log-likelihood function is
\begin{equation}
\log L = \sum_{i=1}^N \log L_i
\end{equation}
\citep[see][p.~2]{bruno04}.

Given that we assumed that $\mu$ follows a normal distribution,
we can calculate the integrals in the log-likelihood function
by the Gauss-Hermite quadrature and then maximise the log-likelihood function
using standard non-linear optimisation algorithms
\citep[see][]{butler82}.

Alternatively, the log-likelihood function can be maximized
using the method of Maximum Simulated Likelihood (MSL),
which allows some flexibility in the specification
of the error terms \citep[p.~799]{greene08}.


\subsubsection{Random effects estimation using the Gauss-Hermite quadrature}

The Gauss-Hermite quadrature is a technique
for approximating specific integrals
with a weighted sum of function values at some specified points.
Applying the Gauss-Hermite quadrature to equation~(\ref{eq:likRandom}),
we get
\begin{align}
L_i  = \frac{1}{\sqrt{\pi}}
   \sum_{h=1}^H w_h
   \Bigg\{ \prod_{t=1}^{T_i} \;&
      \left[
         \pnorm \left( \frac{ a - x_{it} ' \beta - \sqrt{2} \sigma_\mu \psi_h }
            { \sigma_\nu } \right)
      \right]^{I_{it}^a}
      \left[
         \pnorm \left( \frac{ x_{it} ' \beta + \sqrt{2} \sigma_\mu \psi_h - b }
            { \sigma_\nu } \right)
      \right]^{I_{it}^b}
      \label{eq:likRandomGhq}\\
      & \left[
         \frac{1}{\sigma_\nu}
         \dnorm \left( \frac{ y_{it} - x_{it} ' \beta - \sqrt{2} \sigma_\mu \psi_h }
            { \sigma_\nu } \right)
      \right]^{\left( 1 - I_{it}^a - I_{it}^b \right)}
   \Bigg\},  \nonumber
\end{align}
where $H$ is number of quadrature points,
$\psi_1 , \ldots , \psi_H$ are the abscissae, and
$w_1, \ldots , w_H$ are the corresponding weights
\citep[p.~553]{greene08}
 


% \section{Conclusions}
% \label{sec:conclusions}


% \section*{Acknowledgments}

\clearpage
\appendix
\section*{Appendix}

\section{Gradients of the log-likelihood function}
\label{sec:logLikGrad}
\begin{align}
\frac{ \partial \log L }{ \partial \beta_j } =
   \sum_{i = 1}^N \Bigg[ &\;
      - I_i^a \;
         \frac{ \dnorm \left( \frac{ a - x_i ' \beta }{ \sigma } \right) }
            { \pnorm \left( \frac{ a - x_i ' \beta }{ \sigma } \right) }
         \; \frac{ x_{ij} }{ \sigma }
      + I_i^b \;
         \frac{ \dnorm \left( \frac{ x_i ' \beta - b }{ \sigma } \right) }
            { \pnorm \left( \frac{ x_i ' \beta - b }{ \sigma } \right) }
         \; \frac{ x_{ij} }{ \sigma } \\
   & + \left( 1 - I_i^a - I_i^b \right)
      \frac{ y_i - x_i ' \beta }{ \sigma }
      \; \frac{ x_{ij} }{ \sigma }
      \Bigg] \nonumber\\
\frac{ \partial \log L }{ \partial \log \sigma } =
   \sum_{i = 1}^N \Bigg[ \;&
      - I_i^a \;
         \frac{ \dnorm \left( \frac{ a - x_i ' \beta }{ \sigma } \right) }
            { \pnorm \left( \frac{ a - x_i ' \beta }{ \sigma } \right) }
         \; \frac{ a - x_i ' \beta }{ \sigma }
      - I_i^b \;
         \frac{ \dnorm \left( \frac{ x_i ' \beta - b }{ \sigma } \right) }
            { \pnorm \left( \frac{ x_i ' \beta - b }{ \sigma } \right) }
         \; \frac{ x_i ' \beta - b }{ \sigma } \\
   & + \left( 1 - I_i^a - I_i^b \right)
      \left(
         \left( \frac{ y_i - x_i ' \beta }{ \sigma } \right)^2
         - 1
      \right)
      \Bigg] \nonumber
\end{align}


\bibliography{tobit}
% \bibliography{agrarpol}

\end{document}
